% tex
% --------------------------------------------------------------------------------
% Geometric Algorithms Project Report
% This document describes the implementation and analysis of geometric
% algorithms, focusing on line segment intersection and convex hull computation.
% The report details the design choices, experimental setup, results, and discussion.
% Target Users: Developers and researchers in computational geometry.
% --------------------------------------------------------------------------------

\documentclass[a4paper, 10pt, twocolumn]{article}

% Input encoding for non-ASCII characters.
\usepackage[utf8]{inputenc}
% Font encoding to ensure proper font rendering.
\usepackage[T1]{fontenc}
% Graphic inclusion for logos and images.
\usepackage{graphicx}
\usepackage{lipsum}
% Page geometry settings for consistent margins.
\usepackage{geometry}
\usepackage{titlesec}
\usepackage{times}
% Hyperlinks for references.
\usepackage{hyperref}

% Set margins consistently across the document.
\geometry{a4paper, margin=0.75in}
% \geometry{left=2.5cm, right=2.5cm, top=2.5cm, bottom=2.5cm}  % Alternative margin specification

\begin{document}

% --------------------------------------------------------------------------------
% Title Page: Contains project title, logos, author(s), and other vital information.
% --------------------------------------------------------------------------------
\begin{titlepage}
    \centering
    \vspace{3.5cm}
    % Include the university logo.
    \includegraphics[width=0.5\textwidth]{NU-logo.jpg}\par\vspace{1cm}
    % Include the FAST logo.
    \includegraphics[width=0.5\textwidth]{FAST.png}\par\vspace{1cm}
    \vspace{1cm}
    % Main Title of the Project Report.
    {\scshape\Huge\textbf {Geometric Algorithms} \par}
    {\scshape\Huge\textbf {(Line Intersection \& Convex Hull)} \par}
    \vspace{1cm}
    % Subtitle and project report type.
    {\scshape\Large Project Report \par}
    \vspace{1cm}
    % Information about the supervising professor.
    {\scshape\Large Professor Dr. Farrukh Salim (BCS-5E) \par}
    \vspace{1cm}
    % Course and section details.
    {\scshape\Large Design \& Analysis of Algorithms (CS-2009) \par}
    \vspace{1cm}
    % List of group members.
    {\scshape\Large Muhammad Talha (K21-3349) \\ Muhammad Hamza (K21-4579) \\ Muhammad Salar (K21-4619) \par}
    \vfill
    \vspace{1cm}    
    % Institutional information.
    {Foundation of Advancement of Science and Technology \par}
    {National University of Computer and Emerging Sciences \par}
    {Department of Computer Science \par}
    {Karachi, Pakistan \par}
    {Monday, November 27, 2023 \par}
\end{titlepage}

% --------------------------------------------------------------------------------
% Abstract: Summarizes the goals, methodology, and significance of the project.
% --------------------------------------------------------------------------------
\onecolumn
\begin{abstract}
% The abstract communicates the scope and focus areas for the project.
This project focuses on the implementation and analysis of geometric algorithms with varying computational complexities. 
Two main problems are addressed:
\begin{itemize}
    \item determining the intersection of two line segments.
    \item solving the Convex Hull problem using different algorithms.
\end{itemize}
The programming design is implemented in Python and demonstrated using a Jupyter Notebook. 
The experimental setup involves visual demonstration, time complexity measurements, and performance comparisons.
\end{abstract}

\vspace{1.5cm}
\tableofcontents

\twocolumn
% --------------------------------------------------------------------------------
% Introduction: Provides an overview and contextual background for the project.
% --------------------------------------------------------------------------------
\section{Introduction}
% The introduction contextualizes the significance of geometric algorithms in areas such as computer graphics and computational geometry.
Geometric algorithms play a crucial role in various applications, from computer graphics to computational geometry. 
This project delves into two specific problems: line segment intersection and Convex Hull computation. 
The line segment intersection problem explores techniques such as counter-clockwise operations, vector cross products, and the line sweep method; 
while the Convex Hull problem is addressed via brute force, Jarvis-March, Graham scan, Quick Hull, and Monotone Chain methods.

% --------------------------------------------------------------------------------
% Programming Design: Details the design decisions, algorithm choices, and implementation rationale.
% --------------------------------------------------------------------------------
\section{Programming Design}
The implementation is conducted using Python, chosen for its suitability in handling geometric computations. 
This section delves into the thoughtful design choices made, emphasizing clarity and efficiency in the implementation.

\textbf{Line Segment Intersection:} 
The problem is tackled using multiple techniques:
\begin{itemize}
  \item \textit{Counter-clockwise (CCW):} Utilizes the orientation of points for intersection detection.
  \item \textit{Vector Cross Product:} Leverages directional cross products to assess point positions.
  \item \textit{Line Sweep:} Enhances efficiency by sorting events along a sweep line to detect intersecting segments.
\end{itemize}

\textbf{Convex Hull Computation:}
A selection of algorithms is employed, each highlighting different strengths:
\begin{itemize}
  \item \textit{Brute Force:} Provides a baseline check for correctness despite its computational intensity.
  \item \textit{Jarvis-March:} Constructs the hull by selecting points with minimal polar angles.
  \item \textit{Graham Scan:} Efficiently forms the hull by sorting points angularly.
  \item \textit{Quick Hull:} Implements a divide-and-conquer strategy to recursively build hulls.
  \item \textit{Monotone Chain:} Separately computes the upper and lower hulls before merging them.
\end{itemize}

% --------------------------------------------------------------------------------
% Experimental Setup: Explains the testing framework and the method used to measure algorithm performance.
% --------------------------------------------------------------------------------
\section{Experimental Setup}
The program visually renders objects on the screen by selecting random points.
Each algorithm's steps and internal workings are clearly outlined, and the methods used for calculating time and space complexities are described.
This ensures a comprehensive understanding of both the theoretical and practical performance of the solution.

\newpage
% --------------------------------------------------------------------------------
% Results and Discussion: Presents performance data and analyses the strengths and weaknesses of the algorithms.
% --------------------------------------------------------------------------------
\section{Results and Discussion}
Execution times for each algorithm are provided, allowing for a detailed comparison based on their computational complexities. 
The discussion highlights the merits and potential drawbacks of each method, supported by data in the following sections.

\subsection{Line Intersection}
% The table below summarizes the performance metrics for the different line intersection approaches.
\begin{center}
  \centering
  \begin{tabular}{|c|c|}
    \hline
    \textbf {Algorithm} & \textbf {Time Taken} \\
    \hline
    Counter-Clockwise & 0.0000066757\\
    \hline
    Vector Cross Product & 0.0000071526\\
    \hline
    Line Sweep & 0.0023283958\\
    \hline
  \end{tabular}
\end{center}

\subsection{Convex Hull}
% The table below presents the time complexities, execution times, and dataset sizes used for convex hull computation.
\begin{center}
  \centering
  \begin{tabular}{|c|c|c|c|}
    \hline
    \textbf {Algorithm} & \textbf {Time Complexity} & \textbf {Time Taken} & \textbf {Data-set}\\
    \hline %0.0029921532
    Graham Scan & O($nLog(n)$) & $2.992*10^{-3}$ & 500\\
    \hline %5.74589e-05
    Jarvis-March & O($nh$) & $5.746*10^{-5}$ & 500\\
    \hline %0.0036790371
    Quick Hull & O($nLog(n)$) & $3.679*10^{-3}$ & 500\\
    \hline %0.0009784698
    Monotone Chain & O($nLog(n)$) & $9.785*10^{-4}$ & 500\\
    \hline %2264.9023439884
    Brute Force & O($n^3$) & $2.264*10^{3}$ & 500\\
    \hline
  \end{tabular}
\end{center}

% --------------------------------------------------------------------------------
% Conclusion: Summarizes the project findings and outlines future research directions.
% --------------------------------------------------------------------------------
\section{Conclusion}
% The conclusion recaps the key insights and suggests areas for further improvement.
The project concludes by summarizing the key findings and insights gained through the implementation and analysis of geometric algorithms. 
Noteworthy observations include:
\begin{itemize}
    \item \textbf{Unexpected Observations:}\\
    Some algorithms displayed varying performance depending on input data characteristics. For example, Quick Hull showed exceptional speed on well-distributed points but struggled with pathological cases.
    \item \textbf{Challenges Faced:}\\
    Fine-tuning Quick Hull to handle edge cases effectively was challenging. Balancing a divide-and-conquer approach with efficient merging strategies requires additional exploration.
    \item \textbf{Potential Areas for Future Research or Improvements:}\\
    \begin{enumerate}
        \item \textit{Dynamic Input Handling:}\\ 
          Extending the system to work with real-time, dynamic data may improve practical utility. Investigating algorithms that support incremental input could be beneficial.
        \item \textit{Parallelization for Efficiency:}\\ 
          Exploring parallel computing frameworks or algorithms optimized for parallel execution could further enhance performance.
        \item \textit{Hybrid Approaches:}\\ 
          Combining methods such as Jarvis-March and Quick Hull might yield a more robust and efficient convex hull computation solution.
    \end{enumerate}
\end{itemize}

% --------------------------------------------------------------------------------
% References: Provides links and citations for further reading and algorithm background.
% --------------------------------------------------------------------------------
\section{References}
\begin{itemize}
    \item \href{https://en.wikibooks.org/wiki/Algorithm_Implementation/Geometry/Convex_hull/Monotone_chain}{WikiBooks (Monotone Chain Algorithm 1)}
    \item \href{https://www.geeksforgeeks.org/convex-hull-monotone-chain-algorithm/}{GeekForgeeks (Monotone Chain Algorithm 2)}
    \item \href{https://en.wikipedia.org/wiki/Sweep_line_algorithm}{Wikipedia (Line Sweep Algorithm 1)}
    \item \href{https://www.geeksforgeeks.org/closest-pair-of-points-using-sweep-line-algorithm/}{GeekForgeeks (Line Sweep Algorithm 2)}
    \item \href{https://en.wikipedia.org/wiki/Quickhull}{Wikipedia (QuickHull Algorithm 1)}
    \item \href{https://www.geeksforgeeks.org/quickhull-algorithm-convex-hull/}{GeekForgeeks (QuickHull Algorithm 2)}
    \item \href{Introduction to Algorithms by Thomas H. Cormen}{Introduction to Algorithms by Thomas H. Cormen (Reference Book)}
\end{itemize}

\end{document}